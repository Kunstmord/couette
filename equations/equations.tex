\documentclass[12pt]{article}
\usepackage{fontspec}
\setmainfont[Mapping=tex-text]{Times New Roman}
\providecommand{\norm}[1]{\lVert#1\rVert}
\newcommand{\plotWidth}{0.48}

\usepackage{amsmath}
\usepackage{amssymb}
\usepackage{esint}
\usepackage{float}
\usepackage{graphicx}
\usepackage{polyglossia}
\usepackage{color}
\setdefaultlanguage{russian}

\begin{document}


\section{Введение}
\begin{equation}
  \frac{\partial}{\partial x} = \frac{\partial}{\partial z} = \frac{\partial}{\partial t} = 0; \quad \frac{\partial}{\partial y} = \frac{d}{d y},
\end{equation}

\begin{equation}
  \mathbf{v} = (v_{x}, 0, 0),\quad \nabla=\left(0, \frac{\partial}{\partial y}, 0 \right),
\end{equation}

\begin{equation}
  \frac{d A}{d t} = \frac{\partial A}{\partial t} + \mathbf{v} \cdot \nabla A = 0.
\end{equation}

Макропараметры:
\begin{equation}
  n_{c}(y),\:c=1,\ldots,L;\quad T(y);\quad v_{x}(y).
\end{equation}


\begin{equation}
  \mathbf{P} =
  \begin{pmatrix}
  p - p_{rel} & -\eta \frac{\partial v_{x}}{\partial y} & 0 \\
  -\eta \frac{\partial v_{x}}{\partial y} & p - p_{rel} & 0 \\
  0 & 0 & p - p_{rel}
  \end{pmatrix}
\end{equation}

\begin{equation}
  \nabla \mathbf{v} =
  \begin{pmatrix}
  \frac{\partial v_{x}}{\partial x} & \frac{\partial v_{y}}{\partial x} & \frac{\partial v_{z}}{\partial x} \\
  \frac{\partial v_{x}}{\partial y} & \frac{\partial v_{y}}{\partial y} & \frac{\partial v_{z}}{\partial y} \\
  \frac{\partial v_{x}}{\partial z} & \frac{\partial v_{y}}{\partial z} & \frac{\partial v_{z}}{\partial z},
  \end{pmatrix}
  =
  \begin{pmatrix}
  0 & 0 & 0 \\
  \frac{\partial v_{x}}{\partial y} & 0 & 0 \\
  0 & 0 & 0
  \end{pmatrix}
\end{equation}

\begin{equation}
  \mathbf{P} : \nabla \mathbf{v} = \eta \left(\frac{d v_{x}}{d y} \right)^2
\end{equation}

% \begin{equation}
%   \rho \frac{dU}{dt} + \nabla \cdot \mathbf{q} + \mathbf{P} : \nabla \mathbf{v} = 0.
% \end{equation}

\section{Производные основных величин (однотемпературное приближение)}

\begin{equation}
  \mathbf{q} = -\lambda' \nabla T - p \sum_{c} D_{T_{c}} \mathbf{d}_{c} + \sum_{c} \rho_{c} h_{c} \mathbf{V}_{c}.
\end{equation}

\begin{equation}
  d_{c} = \frac{d \frac{n_{c}}{n}}{d y} + \left(\frac{n_{c}}{n} - \frac{\rho_{c}}{\rho} \right) \frac{d \ln p}{d y} =
  \frac{n_{c}'(y)n - n_{c}\sum_{d}n_{d}'(y)}{n^2} + \left(\frac{n_{c}}{n} - \frac{\rho_{c}}{\rho} \right)\frac{1}{p} \frac{d p}{d y},
\end{equation}

\begin{equation}
  \mathbf{V}_{c} = -\sum_{d}D_{cd}\mathbf{d}_{d} - D_{T_{c}}\nabla \ln T,
\end{equation}

\begin{equation}
  \nabla \cdot (\lambda' \nabla T) = \frac{d (\lambda' T'(y))}{dy} = \lambda' T''(y) + T'(y) \frac{d \lambda'}{d y} = \lambda' T''(y) + T'(y) \left(T'(y) \frac{\partial \lambda'}{\partial T} + \sum_{c} n_{c}'(y) \frac{\partial \lambda'}{\partial n_{c}} \right),\label{d-lambda-nabla-T}
\end{equation}

\begin{equation}
  \nabla \left(p \sum_{c} D_{T_{c}} \mathbf{d}_{c} \right) = \frac{d p}{d y} \sum_{c} D_{T_{c}} d_{c} + p \left(\sum_{c} D_{T_{c}} \frac{d d_{c}}{d y} + \sum_{c} \frac{d D_{T_{c}}}{d y} d_{c} \right),
\end{equation}

\begin{equation}
  \frac{d p}{d y} = p'(y) = k \left(T\sum_{c} n_{c}'(y) +  T'(y)n \right),
\end{equation}

\begin{equation}
  p''(y) = k \left(2T'(y)\sum_{c} n_{c}'(y) + T\sum_{c} n_{c}''(y) +  T''(y)n\right),
\end{equation}

\begin{multline}
  \frac{d}{dy}d_{c} = \frac{n_{c}''(y)}{n} - 2\frac{n_{c}'(y)\sum_{d}n_{d}'(y)}{n^2} - \frac{n_{c}\sum_{d}n_{d}''(y)}{n^2} + 2\frac{n_{c}\left(\sum_{d}n_{d}'(y) \right)^2}{n^3} + \\
  + \frac{n_{c}'(y)n - n_{c}\sum_{d}n_{d}'(y)}{n^2} \frac{1}{p} \frac{dp}{dy} - \frac{n_{c}}{n} \frac{p'(y)^2}{p^2} + \frac{n_{c}}{n} \frac{1}{p}p''(y) + \\
  + m_{c} \left(\frac{n_{c}'(y)\rho - n_{c}\sum_{d}m_{d}n_{d}'(y)}{\rho^2} \frac{1}{p} \frac{dp}{dy} - \frac{n_{c}}{\rho} \frac{p'(y)^2}{p^2} + \frac{n_{c}}{\rho} \frac{1}{p}p''(y) \right)
\end{multline}

\begin{equation}
  \frac{d D_{T_{c}}}{d y} = T'(y) \frac{\partial D_{T_{c}}}{\partial T} + \sum_{d} \frac{\partial D_{T_{c}}}{\partial n_{d}} n_{d}'(y),
\end{equation}

\begin{equation}
  \frac{d h_{c}}{d y} = \frac{5}{2} \frac{k}{m_{c}}T'(y) + \frac{\partial \left(E_{rot,c} + E_{vibr,c} \right)}{\partial T}T'(y),
\end{equation}

\begin{equation}
  \frac{d V_{c}}{dy} = -\frac{d \sum_{d}D_{cd}d_{d}}{dy} - \frac{d D_{T_{c}}}{d y} \frac{1}{T} T'(y) - D_{T_{c}} \frac{T(y)T''(y) - T'(y)T'(y)}{T(y)^2},
\end{equation}

\begin{equation}
  \frac{d D_{cd}}{dy} = \sum_{f}\frac{\partial D_{cd}}{\partial n_{f}}n_{f}'(y) + \frac{\partial D_{cd}}{\partial T}T'(y),
\end{equation}

\section{Уравнения переноса}

\begin{equation}
  \frac{d n_{c}}{d y} V_{c} + n_{c}\frac{d V_{c}}{d y} = R_{c}^{react},\quad c=1,\ldots,L,\label{densityeqn}
\end{equation}

\begin{equation}
  \frac{d}{d y}\left(\eta \frac{d v_{x}}{d y} \right) = 0,\label{motioneqn}
\end{equation}

\begin{equation}
  \frac{d q}{d y} - \eta \left(\frac{d v_{x}}{d y} \right)^2 = 0.\label{energyeqn}
\end{equation}

\subsection{Однокомпонентная смесь без внутренних степеней свободы}
Рассмотрим однокомпонентное течение без внутренних степенй свободы.
В таком течении $V_{c}$ = 0, $\frac{\partial n}{\partial y} = 0$, $\eta=\eta(T)$, $\lambda'=\lambda'(T)$.

% Энтальпия для такого течения вычисляется по формуле
% \begin{equation}
%   mh = \frac{5}{2}kT + \varepsilon_{c}.
% \end{equation}

Уравнение теплопроводности принимает вид:

% \begin{multline}
%   \eta \left(\frac{d v_{x}}{d y} \right)^2 -\lambda' T'' - T' \left(T' \frac{\partial \lambda'}{\partial T} + n' \frac{\partial \lambda'}{\partial n} \right)
%   + n' m h V + \frac{5}{2}nkT' V + \\
%   + nmh \left(- \left(T' \frac{\partial D_{T}}{\partial T} + \frac{\partial D_{T}}{\partial n} n' \right) \frac{1}{T} T' - D_{T} \frac{TT'' - T'^2}{T^2} \right)=0
% \end{multline}

Систему уравнений (\ref{densityeqn}) -- (\ref{energyeqn}) можно свести к системе ОДУ первого порядка:

% \begin{equation}
%   D_{T} T' \frac{d n}{d y} + n\left(\left(T' \frac{\partial D_{T}}{\partial T} + \frac{\partial D_{T}}{\partial n} n'(y)\right)
%    T' + D_{T} \frac{T dT' / dy - T'^2}{T} \right) = 0
% \end{equation}

\begin{equation}
  \frac{d v_{x}}{d y} = v_{x}',\label{eqn-dimensional-dv}
\end{equation}

\begin{equation}
  \frac{d T}{d y} = T',\label{eqn-dimensional-dT}
\end{equation}

\begin{equation}
  \frac{d v_{x}'}{d y} = -\frac{1}{\eta} v_{x}' T' \frac{\partial \eta}{\partial T},\label{eqn-dimensional-d2v}
\end{equation}

\begin{equation}
   \frac{dT'}{dy} = -\frac{\eta v_{x}'^2}{\lambda'}
  - \frac{1}{\lambda'} T'^2 \frac{\partial \lambda'}{\partial T}.\label{eqn-dimensional-d2T}
\end{equation}

Граничные условия имеют вид

\begin{equation}
  v_{x}(0) = 0,\:v_{x}(Y) = v_{wall},\:T(0) = T_{wall,1},\:T(Y) = T_{wall,2}.
\end{equation}
где $Y$ --- высота канала.

В случае, если $\frac{\partial \lambda'}{\partial T} = \frac{\partial \eta}{\partial T} = 0$, то задача имеет аналитическое решение:

\begin{equation}
  v_{x}(y) = v_{wall}\frac{y}{Y},
\end{equation}

\textcolor{red}{Надо проверить/исправить}
\begin{equation}
  T(y) = \frac{\eta v_{x}'^2}{2\lambda'}y^2 + C_{1}y + C_{2} = \frac{\eta v_{wall}^2}{2\lambda'}\left(\frac{y}{Y}\right)^2 + C_{1}y + T_{wall,1},
\end{equation}
где

\begin{equation}
  C_{1}= \frac{T_{wall,2} - T_{wall,1} -\frac{\eta v_{x}'^2}{2\lambda'}}{Y}.
\end{equation}

Для чистого атомарного газа коэффициенты теплопроводности $\lambda'$ и сдвиговой вязкости $\eta$ вычисляются следующим образом:

\begin{equation}
  \lambda' = \frac{75k^2T}{32m\Omega^{(2,2)}},
\end{equation}

\begin{equation}
  \eta = \frac{5kT}{8\Omega^{(2,2)}}.
\end{equation}

Систему (\ref{eqn-dimensional-dv} -- \ref{eqn-dimensional-d2T}) можно привести к безразмерному виду, введя следующие переменные:

\begin{equation}
  \zeta = \frac{y}{Y},\quad x_{1} = \frac{v_{x}}{v_{wall}},\quad x_{2} = \frac{T}{T_{wall, 2}}.
\end{equation}

Тогда $\frac{\partial }{\partial y}=\frac{\partial }{\partial \zeta}\frac{1}{Y}$, $\frac{\partial }{\partial T}=\frac{\partial }{\partial x_{2}}\frac{1}{T_{wall,2}}$, и обезразмеренная система (\ref{eqn-dimensional-dv} -- \ref{eqn-dimensional-d2T}) имеет вид

\begin{equation}
  \frac{d x_{1}}{d \zeta} = Y x_{1}',\label{eqn-dimensionless-dx1}
\end{equation}

\begin{equation}
  \frac{d x_{2}}{d \zeta} = Y x_{2}',\label{eqn-dimensionless-dx2}
\end{equation}

\begin{equation}
  \frac{d x_{1}'}{d \zeta} = -\frac{Y}{\eta}x_{1}'x_{2}'\frac{\partial \eta}{\partial x_{2}},\label{eqn-dimensionless-dx21}
\end{equation}

\begin{equation}
  \frac{d x_{2}'}{d \zeta} = -\frac{Y}{\lambda'}\left(\frac{\eta}{T_{wall,2}}v_{wall}^{2}x_{1}'^2+x_{2}'^{2}\frac{\partial \lambda'}{\partial x_{2}} \right).
\end{equation}

\subsection{Расчеты для чистого аргона}
Предполагалось отсутствие скачка температуры у стенки, а скорость скольжения выражалась по формуле:

\begin{equation}
  v_{slip} = \sigma_{P} \frac{\eta v_{m}}{nkT}\frac{d v_{x}}{d \nu} + \sigma_{T}\frac{\eta}{\rho}\frac{d \ln T}{d \nu},
\end{equation}
где $v_{m} = \sqrt{2kT_{wall} / m}$, $\nu$ --- нормаль к поверхности; для нижней стенки ($y=0$) $d / d \nu$ = $d / d y$, для верхней ($y=Y$) $d / d \nu$ = $-d / d y$.

Для чисто диффузного отражения коэффициенты равны $\sigma_{P} = 0.8862$, $\sigma_{T} = 0.75$.

Значение численной плотности смеси бралось равным $n=1.4 \cdot 10^{20}$.

Случай 1 (температуры стенок равны 1000K):
\begin{center}
\includegraphics[width=10cm]{plots/vel_tc3}
\end{center}
\begin{center}
\includegraphics[width=10cm]{plots/T_tc3}
\end{center}
\begin{center}
\includegraphics[width=10cm]{plots/eta_tc3}
\end{center}

Случай 2 (температуры стенок равны 273K):
\begin{center}
\includegraphics[width=10cm]{plots/vel_tc1}
\end{center}
\begin{center}
\includegraphics[width=10cm]{plots/T_tc1}
\end{center}
\begin{center}
\includegraphics[width=10cm]{plots/eta_tc1}
\end{center}

\section{Производные основных величин (поуровневое приближение)}
Рассмотрим течение чисто молекулярного газа в поуровневом приближении.

Уравнения для численных плотностей имеют вид
\begin{equation}
  \frac{d n_{i}}{d y} V_{i} + n_{i}\frac{d V_{i}}{d y} = R_{i}^{vibr},\quad c=1,\ldots,L.\label{densityeqn-sts}
\end{equation}

Скорость диффузии:

% \begin{equation}
%    \mathbf{V}_{i} = -D_{ii}\mathbf{d}_{i} -D \sum_{k \neq i}\mathbf{d}_{k} - D_{T}\nabla \ln T,
% \end{equation}

% тепловой поток:

% \begin{equation}
%   \mathbf{q} = -\lambda' \nabla T - p D_{T} \mathbf{d} + \sum_{i} \left(\frac{5}{2}kT + \left<\varepsilon^{i}_{j} \right>_{rot} + \varepsilon_{i} + \varepsilon \right) n_{i} \mathbf{V}_{i}.
% \end{equation}

\begin{equation}
   \mathbf{V}_{i} = -D_{ii}\mathbf{d}_{i} -D \sum_{k \neq i}\mathbf{d}_{k},
\end{equation}

тепловой поток:

\begin{equation}
  \mathbf{q} = -\lambda' \nabla T + \sum_{i} \left(\frac{5}{2}kT + \left<\varepsilon^{i}_{j} \right>_{rot} + \varepsilon_{i} + \varepsilon \right) n_{i} \mathbf{V}_{i}.
\end{equation}

Основываясь на упрощенных формулах для коэффициентов диффузии, и введя следующую величину:

\begin{equation}
  \mathcal{D} = \frac{3kT}{8nm}\frac{1}{\Omega^{(1,1)}},
\end{equation}

имеем выражения для $D_{ii}$ и $D$:

\begin{equation}
  D_{ii} = \mathcal{D} \left(\frac{\rho}{\rho_{i}} - 1 \right)=\mathcal{D} \left(\frac{n}{n_{i}} - 1 \right),\quad D = -\mathcal{D}.
\end{equation}

В силу того, что смесь является однокомпонентной, выражение для диффузионных термодинамических сил упрощается и принимает вид

\begin{equation}
  \mathbf{d}_{i} = \nabla \left(\frac{n_{i}}{n} \right).
\end{equation}

С учетом того, что течение одномерное, имеем

\begin{equation}
  d_{i} = \frac{n_{i}'(y)n - n_{i}n'(y)}{n^2},
\end{equation}

\begin{equation}
  \frac{d d_{i}}{d y} = \frac{n_{i}''(y)}{n} - 2\frac{n_{i}'(y)n'(y)}{n^2} - \frac{n_{i}n''(y)}{n^2} + 2\frac{n_{i}\left(n'(y) \right)^2}{n^3},
\end{equation}

\begin{equation}
  \frac{d\mathcal{D}}{dy} = \frac{\partial \mathcal{D}}{\partial T} T'(y) + \frac{\partial \mathcal{D}}{\partial n} n'(y),
\end{equation}

\begin{equation}
  \frac{d D_{ii}}{dy} = \left(\frac{\partial \mathcal{D}}{\partial T} \frac{d T}{d y} + \frac{\partial \mathcal{D}}{\partial n} n'(y) \right)
  \left(\frac{n}{n_{i}} - 1 \right) + \mathcal{D} \frac{n'(y)n_{i} - n_{i}'(y)n}{n_{i}^2}.
\end{equation}

Введем обозначение

\begin{equation}
  n_{k,i} = \sum_{k \neq i} n_{k}(y).
\end{equation}

Тогда величину $\sum_{k \neq i} d d_{k} / dy$ можно преобразовать:

\begin{multline}
  \sum_{k \neq i} \frac{d d_{k}}{d y} = \sum_{k \neq i} \left(\frac{n_{i}''(y)}{n} - 2\frac{n_{i}'(y)n'(y)}{n^2} - \frac{n_{i}n''(y)}{n^2} + 2\frac{n_{i}\left(n'(y) \right)^2}{n^3}\right) = \\
  = \frac{n_{k,i}''(y)}{n} - 2\frac{n_{k,i}'(y)n'(y)}{n^2} - \frac{n_{k,i}'(y) n''(y)}{n^2} + 2\frac{n_{k,i}\left(n'(y) \right)^2}{n^3}.
\end{multline}

Производная скорости диффузии равна

\begin{multline}
  \frac{dV_{i}}{dy} = -\left[\left\{\left(\frac{\partial \mathcal{D}}{\partial n}n' + \frac{\partial \mathcal{D}}{\partial T}T' \right)\left(\frac{n}{n_{i}} - 1 \right) 
  + \mathcal{D} \frac{n'n_{i} - n n_{i}'}{n_{i}^2} \right\} \frac{n_{i}'n - n_{i}n'}{n^2} \right. + \\
  + \mathcal{D} \left(\frac{n}{n_{i}} - 1\right) \left(\frac{n_{i}''}{n} - 2\frac{n_{i}'n'}{n^2} - \frac{n_{i}n''}{n^2} + 2\frac{n_{i}\left(n' \right)^2}{n^3} \right) - \\
  - \left(\frac{\partial \mathcal{D}}{\partial n}n' + \frac{\partial \mathcal{D}}{\partial T}T' \right) \frac{n_{k,i}'n - n_{k,i}n'}{n^2} 
  - \left.\mathcal{D} \left(\frac{n_{k,i}''}{n} - 2\frac{n_{k,i}'n'}{n^2} - \frac{n_{k,i} n''}{n^2} + 2\frac{n_{k,i}\left(n' \right)^2}{n^3} \right) \right].
\end{multline}

Уравнения, описывающие изменение численных плотностей, принимают вид

\begin{multline}
  -n_{i}' \left(\frac{n_{i}'n - n_{i}n'}{n^2} \mathcal{D} \left(\frac{n}{n_{i}} - 1 \right) - \mathcal{D} \frac{n_{k,i}'n - n_{k,i}n'}{n^2}\right) - \\  % no thermodiffusion
  - n_{i} \left[\left\{\left(\frac{\partial \mathcal{D}}{\partial n}n' + \frac{\partial \mathcal{D}}{\partial T}T' \right)\left(\frac{n}{n_{i}} - 1 \right) 
  + \mathcal{D} \frac{n'n_{i} - n n_{i}'}{n_{i}^2} \right\} \frac{n_{i}'n - n_{i}n'}{n^2} \right. + \\
  + \mathcal{D} \left(\frac{n}{n_{i}} - 1\right) \left(\frac{n_{i}''}{n} - 2\frac{n_{i}'n'}{n^2} - \frac{n_{i}n''}{n^2} + 2\frac{n_{i}\left(n' \right)^2}{n^3} \right) - \\
  - \left(\frac{\partial \mathcal{D}}{\partial n}n' + \frac{\partial \mathcal{D}}{\partial T}T' \right) \frac{n_{k,i}'n - n_{k,i}n'}{n^2} - \\
  - \left.\mathcal{D} \left(\frac{n_{k,i}''}{n} - 2\frac{n_{k,i}'n'}{n^2} - \frac{n_{k,i} n''}{n^2} + 2\frac{n_{k,i}\left(n' \right)^2}{n^3} \right) \right]
  = R_{i}^{vibr},\quad i=1,\ldots L. \label{system-ni-sts}
\end{multline}

Видно, что в уравнение изменения численной плотности молекул на $i$-м колебательном уровне входят вторые производные от численных плотностей всех остальных колебательных уровней, поэтому систему уравнений (\ref{system-ni-sts}) имеет смысл для наглядности записать в матричном виде, т.к. для численного решения систему ОДУ на каждом шаге сначала придется решать линейную систему для нахождения значений $n_{i}''$.

В матричном виде система выглядит следующим образом:

\begin{equation}
  \mathcal{D} A \cdot N = b,
\end{equation}
где $N=\left(n_{0}'', n_{1}'', \ldots, n_{L}''\right)$, $i$-ая компонента вектора $b$ определяется следующей формулой \textcolor{red}{(наверное, можно еще упростить, объединив $n_{k,i}$ и $n_{i}$, где возможно)}:

\begin{multline}
  b_{i} = n_{i}' \mathcal{D} \frac{n_{i}'n - n_{i}n'}{n n_{i}} + n_{i} \left[\left\{\left(\frac{\partial \mathcal{D}}{\partial n}n' + \frac{\partial \mathcal{D}}{\partial T}T' \right)\left(\frac{n}{n_{i}} - 1 \right) 
  + \mathcal{D} \frac{n'n_{i} + n n_{i}'}{n_{i}^2} \right\} \frac{n_{i}'n - n_{i}n'}{n^2} \right. + \\
  + \mathcal{D} \left(\frac{n}{n_{i}} - 1\right) \left( - 2\frac{n_{i}'n'}{n^2} + 2\frac{n_{i}\left(n' \right)^2}{n^3} \right)
  - \left(\frac{\partial \mathcal{D}}{\partial n}n' + \frac{\partial \mathcal{D}}{\partial T}T' \right) \frac{n_{k,i}'n - n_{k,i}n'}{n^2} - \\
  - \left.\mathcal{D} \left(- 2\frac{n_{k,i}'n'}{n^2}+ 2\frac{n_{k,i}\left(n' \right)^2}{n^3} \right) \right] + R_{i}^{vibr},
\end{multline}
а компонент $a_{ij}$ ($i$-ая строка, $j$-й столбец) матрицы $A$ задается выражением

\begin{equation}
  a_{ij} = \frac{n_{i}}{n} - \delta_{ij}.
\end{equation}

% \begin{equation}
%   \frac{d n}{d y} = n \frac{T'^2 \frac{\partial D_{T}}{\partial T} + D_{T} \frac{T dT'/dy - T'^2}{T}}{T' \left(D_{T} + n \frac{\partial D_{T}}{\partial n} \right)},
% \end{equation}

% \begin{equation}
%   \frac{d v_{x}'}{d y} = -\frac{1}{\eta} v_{x}' \left(T' \frac{\partial \eta}{\partial T} + n' \frac{\partial \eta}{\partial n} \right),
% \end{equation}

% \begin{multline}
%    \frac{dT'}{dy} = \frac{\eta v_{x}'^2}{\lambda' + D_{T} n \tau(T) }
%   - \frac{T'}{\lambda' + D_{T} n\tau(T)} \left(T' \frac{\partial \lambda'}{\partial T} + n' \frac{\partial \lambda'}{\partial n} + \right. \\ + \left. n'\tau(T)D_{T} + \frac{5}{2}nkD_{T}\frac{T'}{T}
%   + n\tau(T) \left(T' \frac{\partial D_{T}}{\partial T} + n' \frac{\partial D_{T}}{\partial n} + D_{T} \frac{T'}{T} \right)\right),
% \end{multline}
% где

% \begin{equation}
%   \tau(T) = \frac{5}{2}k + \frac{\varepsilon}{T}.
% \end{equation}

\end{document}
